
% Arxiv styling
% Based on https://github.com/kourgeorge/arxiv-style
\documentclass{article}
\usepackage{arxiv}

\usepackage[utf8]{inputenc} % allow utf-8 input
\usepackage[T1]{fontenc}    % use 8-bit T1 fonts
\usepackage{hyperref}       % hyperlinks
\usepackage{url}            % simple URL typesetting
\usepackage{booktabs}       % professional-quality tables
\usepackage{amsfonts}       % blackboard math symbols
\usepackage{nicefrac}       % compact symbols for 1/2, etc.
\usepackage{microtype}      % microtypography
\usepackage{floatrow}
%\usepackage{natbib}
\usepackage{doi}
\usepackage{graphicx}
\usepackage{subcaption}

% Document
\begin{document}

%% Title
\title{(preliminary title) Leaf compression for tiny Random Forests}

%% Author info
\author{
    Jon Nordby \\
	Soundsensing AS \\
	\texttt{jon@soundsensing.no} \\	
}

% Remove the date
\date{}

\maketitle
\renewcommand{\abstractname}{\vspace{-\baselineskip}} % erase the space between authors and abstract

%% Abstract
\begin{abstract}	\noindent

% Keywords. Maximum 5
\noindent \textbf{Keywords}: Random Forest, TinyML, model compression, tree-based ensemble

\end{abstract}


\section{Introduction}

Big picture motivation.

\noindent

\newpage
\section{Background}

TinyML setting. On device inference. Low power. Wireless sensing.
Success cases for Random Forest in TinyML.
Human Activity Recognition. \cite{elsts_are_2021}
Animal behavior. \cite{tatler_high_2018} \cite{kleanthous_feature_2020} \cite{tran_iot-based_2022}
Battery monitoring. \cite{mawonou_state--health_2021}

Structure of a random forest.
Decision nodes, leaf nodes.
Voting rules. argmax / Hard majority.
Probability average. Soft.

Fully grown versus partially grown trees.

\subsection{Related work}

Optimization techiques for decision trees and decision tree ensembles like Random Forest.
Integer quantization of features.
FPGA/hardware related.
SIMD tree evaluation.

Optimization for microcontrollers \cite{tabanelli_optimizing_2022}
Feature selection. \cite{uddin_guided_2015}, \cite{elsts_energy-efficient_2020}
Early stopping. \cite{daghero_low-overhead_2022} \cite{daghero_adaptive_2021}
Parallelization.
Cache-awareness.

Software tools for deploying RF to TinyML. Open-source


\section{Methods}

What we cover: Feature quantization. Leaf quantization. Leaf clustering.



OpenML CC18 datasets.
Selected / eliminated datasets.
TODO: list exact identifiers for that which was eliminated

Train a general model. Using scikit-learn.

Apply compression strategy to trained model. Re-running predictions on test set.

Cross validation.

? Demonstrator of performance on one or few TinyML datasets.
PAMAP.
Comparisons with other frameworks. 

\footnote{The code for reproducing the experiments can be found at https://github.com/jonnor/leaf-quantization}

\newpage
\section{Results}

Performance with 16bit and 8 bit feature quantization, vs floating point.


Percentage of model space (bytes) used for leaves.
Histogram over models from all the datasets. Using 16 bit 
For float (no quantization), uint16 and uint8 quantization (with deduplication).
Determines the saving potential we have.

? claim. Model size is the typical main contraint for RF in TinyML settings. Execution time is less of a problem

%% DISCUSSION
\section{Discussion}

\subsection{Further work}

Leaf clustering for regression.


%\newpage
\section{Conclusions}


\section*{Acknowledgements}
\noindent

Martin Stensgård

% References
\newpage
\bibliographystyle{unsrt}
\bibliography{references.bib} 

\end{document}

